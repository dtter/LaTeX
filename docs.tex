\documentclass{article}

\title{Latex Documentation}
\author{}
\date{}

\begin{document}

\maketitle

\section{Class}
\begin{center}
    \verb!\documentclass[option1, option2, etc.]{article}!
\end{center}

\begin{itemize}
    \item Font size
    \item Paper size and format:
    \begin{itemize}
        \item \verb!a4paper!
        \item \verb!letterpaper!
        \item \verb!a5paper!
        \item \verb!b5paper!
        \item \verb!executivepaper!
        \item \verb!legalpaper!
    \end{itemize}
    \item Column: \verb!onecolumn, twocolumn!
    \item Formula-specific options:
    \begin{itemize}
        \item \verb!fleqn!: left-alignemnt formulas
        \item \verb!leqno!: labels formulas on the left-hand side
    \end{itemize}
    \item Titlepage
    \begin{itemize}
        \item \verb!notitlepage!
        \item \verb!titlepage!: title for every page
    \end{itemize}
    \item Chapter opening
    \begin{itemize}
        \item \verb!openany!
        \item \verb!openright!: the new chapter always starts on the right
    \end{itemize}
\end{itemize}

\section{Environments}
Standard tools used to accomplish typesetting tasks in \LaTeX:\\
\verb!\begin{%env name}...\end{%env name}!

\section{Sectioning}
\begin{verbatim}
    \section{...}
    \subsection{...}
    \subsubsection{...}
\end{verbatim}

\section{Math modes}
\subsection{In-line }
    \verb|$...$| or \verb|\(...\)|
\subsubsection{Adding displaystyle}
    \verb!$\displaystyle ...$!
    \begin{itemize}
        \item Example:\\
    This is normal size in-line equation: $\gamma=\frac{1}{\sqrt{1-(\frac{v}{c})^{2}}}$. This is the \verb!displaystyle!d in-line equation: $\displaystyle \gamma=\frac{1}{\sqrt{1-(\frac{v}{c})^{2}}}$
    \end{itemize}


\subsection{Display/Centered}
\begin{center}
    \verb|$$...$$|
\end{center}
\section{Inserting \LaTeX Code}
    \verb|\verb!...!| or \verb!\verb|...|! \\or

\begin{verbatim}
    \begin{verbatim}
        ... multiline code
    \end{...
\end{verbatim}

% --- NEW CHAPTER 
\newpage
\Large Packages
\normalsize

\begin{center}
    \begin{tabular}{c|c}
    \hline
        amsmath & desc \\
        \hline
        amssymb & desc \\
        \hline
        amsthm & desc \\

    \end{tabular}
\end{center}

\end{document}
